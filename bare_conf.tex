
%% bare_conf.tex
%% V1.4b
%% 2015/08/26
%% by Michael Shell
%% See:
%% http://www.michaelshell.org/
%% for current contact information.
%%
%% This is a skeleton file demonstrating the use of IEEEtran.cls
%% (requires IEEEtran.cls version 1.8b or later) with an IEEE
%% conference paper.
%%
%% Support sites:
%% http://www.michaelshell.org/tex/ieeetran/
%% http://www.ctan.org/pkg/ieeetran
%% and
%% http://www.ieee.org/

%%*************************************************************************
%% Legal Notice:
%% This code is offered as-is without any warranty either expressed or
%% implied; without even the implied warranty of MERCHANTABILITY or
%% FITNESS FOR A PARTICULAR PURPOSE!
%% User assumes all risk.
%% In no event shall the IEEE or any contributor to this code be liable for
%% any damages or losses, including, but not limited to, incidental,
%% consequential, or any other damages, resulting from the use or misuse
%% of any information contained here.
%%
%% All comments are the opinions of their respective authors and are not
%% necessarily endorsed by the IEEE.
%%
%% This work is distributed under the LaTeX Project Public License (LPPL)
%% ( http://www.latex-project.org/ ) version 1.3, and may be freely used,
%% distributed and modified. A copy of the LPPL, version 1.3, is included
%% in the base LaTeX documentation of all distributions of LaTeX released
%% 2003/12/01 or later.
%% Retain all contribution notices and credits.
%% ** Modified files should be clearly indicated as such, including  **
%% ** renaming them and changing author support contact information. **
%%*************************************************************************


% *** Authors should verify (and, if needed, correct) their LaTeX system  ***
% *** with the testflow diagnostic prior to trusting their LaTeX platform ***
% *** with production work. The IEEE's font choices and paper sizes can   ***
% *** trigger bugs that do not appear when using other class files.       ***                          ***
% The testflow support page is at:
% http://www.michaelshell.org/tex/testflow/



\documentclass[conference]{IEEEtran}
% Some Computer Society conferences also require the compsoc mode option,
% but others use the standard conference format.
%
% If IEEEtran.cls has not been installed into the LaTeX system files,
% manually specify the path to it like:
% \documentclass[conference]{../sty/IEEEtran}





% Some very useful LaTeX packages include:
% (uncomment the ones you want to load)


% *** MISC UTILITY PACKAGES ***
%
%\usepackage{ifpdf}
% Heiko Oberdiek's ifpdf.sty is very useful if you need conditional
% compilation based on whether the output is pdf or dvi.
% usage:
% \ifpdf
%   % pdf code
% \else
%   % dvi code
% \fi
% The latest version of ifpdf.sty can be obtained from:
% http://www.ctan.org/pkg/ifpdf
% Also, note that IEEEtran.cls V1.7 and later provides a builtin
% \ifCLASSINFOpdf conditional that works the same way.
% When switching from latex to pdflatex and vice-versa, the compiler may
% have to be run twice to clear warning/error messages.






% *** CITATION PACKAGES ***
%
%\usepackage{cite}
% cite.sty was written by Donald Arseneau
% V1.6 and later of IEEEtran pre-defines the format of the cite.sty package
% \cite{} output to follow that of the IEEE. Loading the cite package will
% result in citation numbers being automatically sorted and properly
% "compressed/ranged". e.g., [1], [9], [2], [7], [5], [6] without using
% cite.sty will become [1], [2], [5]--[7], [9] using cite.sty. cite.sty's
% \cite will automatically add leading space, if needed. Use cite.sty's
% noadjust option (cite.sty V3.8 and later) if you want to turn this off
% such as if a citation ever needs to be enclosed in parenthesis.
% cite.sty is already installed on most LaTeX systems. Be sure and use
% version 5.0 (2009-03-20) and later if using hyperref.sty.
% The latest version can be obtained at:
% http://www.ctan.org/pkg/cite
% The documentation is contained in the cite.sty file itself.






% *** GRAPHICS RELATED PACKAGES ***
%
\ifCLASSINFOpdf
  \usepackage[pdftex]{graphicx}
  % declare the path(s) where your graphic files are
  \graphicspath{{./images/}}
  % and their extensions so you won't have to specify these with
  % every instance of \includegraphics
  \DeclareGraphicsExtensions{.pdf,.jpeg,.png}
\else
  % or other class option (dvipsone, dvipdf, if not using dvips). graphicx
  % will default to the driver specified in the system graphics.cfg if no
  % driver is specified.
  % \usepackage[dvips]{graphicx}
  % declare the path(s) where your graphic files are
  % \graphicspath{{../eps/}}
  % and their extensions so you won't have to specify these with
  % every instance of \includegraphics
  % \DeclareGraphicsExtensions{.eps}
\fi
% graphicx was written by David Carlisle and Sebastian Rahtz. It is
% required if you want graphics, photos, etc. graphicx.sty is already
% installed on most LaTeX systems. The latest version and documentation
% can be obtained at:
% http://www.ctan.org/pkg/graphicx
% Another good source of documentation is "Using Imported Graphics in
% LaTeX2e" by Keith Reckdahl which can be found at:
% http://www.ctan.org/pkg/epslatex
%
% latex, and pdflatex in dvi mode, support graphics in encapsulated
% postscript (.eps) format. pdflatex in pdf mode supports graphics
% in .pdf, .jpeg, .png and .mps (metapost) formats. Users should ensure
% that all non-photo figures use a vector format (.eps, .pdf, .mps) and
% not a bitmapped formats (.jpeg, .png). The IEEE frowns on bitmapped formats
% which can result in "jaggedy"/blurry rendering of lines and letters as
% well as large increases in file sizes.
%
% You can find documentation about the pdfTeX application at:
% http://www.tug.org/applications/pdftex

\usepackage[pdftex,linkcolor=blue,citecolor=blue,backref=page]{hyperref}

\usepackage{algorithm}

\usepackage{algorithmic}
\usepackage{url}

% *** MATH PACKAGES ***
%
\usepackage{amsmath}
% A popular package from the American Mathematical Society that provides
% many useful and powerful commands for dealing with mathematics.
%
% Note that the amsmath package sets \interdisplaylinepenalty to 10000
% thus preventing page breaks from occurring within multiline equations. Use:
%\interdisplaylinepenalty=2500
% after loading amsmath to restore such page breaks as IEEEtran.cls normally
% does. amsmath.sty is already installed on most LaTeX systems. The latest
% version and documentation can be obtained at:
% http://www.ctan.org/pkg/amsmath





% *** SPECIALIZED LIST PACKAGES ***
%
%\usepackage{algorithmic}
% algorithmic.sty was written by Peter Williams and Rogerio Brito.
% This package provides an algorithmic environment fo describing algorithms.
% You can use the algorithmic environment in-text or within a figure
% environment to provide for a floating algorithm. Do NOT use the algorithm
% floating environment provided by algorithm.sty (by the same authors) or
% algorithm2e.sty (by Christophe Fiorio) as the IEEE does not use dedicated
% algorithm float types and packages that provide these will not provide
% correct IEEE style captions. The latest version and documentation of
% algorithmic.sty can be obtained at:
% http://www.ctan.org/pkg/algorithms
% Also of interest may be the (relatively newer and more customizable)
% algorithmicx.sty package by Szasz Janos:
% http://www.ctan.org/pkg/algorithmicx




% *** ALIGNMENT PACKAGES ***
%
\usepackage{array}
% Frank Mittelbach's and David Carlisle's array.sty patches and improves
% the standard LaTeX2e array and tabular environments to provide better
% appearance and additional user controls. As the default LaTeX2e table
% generation code is lacking to the point of almost being broken with
% respect to the quality of the end results, all users are strongly
% advised to use an enhanced (at the very least that provided by array.sty)
% set of table tools. array.sty is already installed on most systems. The
% latest version and documentation can be obtained at:
% http://www.ctan.org/pkg/array


% IEEEtran contains the IEEEeqnarray family of commands that can be used to
% generate multiline equations as well as matrices, tables, etc., of high
% quality.




% *** SUBFIGURE PACKAGES ***
%\ifCLASSOPTIONcompsoc
%  \usepackage[caption=false,font=normalsize,labelfont=sf,textfont=sf]{subfig}
%\else
%  \usepackage[caption=false,font=footnotesize]{subfig}
%\fi
% subfig.sty, written by Steven Douglas Cochran, is the modern replacement
% for subfigure.sty, the latter of which is no longer maintained and is
% incompatible with some LaTeX packages including fixltx2e. However,
% subfig.sty requires and automatically loads Axel Sommerfeldt's caption.sty
% which will override IEEEtran.cls' handling of captions and this will result
% in non-IEEE style figure/table captions. To prevent this problem, be sure
% and invoke subfig.sty's "caption=false" package option (available since
% subfig.sty version 1.3, 2005/06/28) as this is will preserve IEEEtran.cls
% handling of captions.
% Note that the Computer Society format requires a larger sans serif font
% than the serif footnote size font used in traditional IEEE formatting
% and thus the need to invoke different subfig.sty package options depending
% on whether compsoc mode has been enabled.
%
% The latest version and documentation of subfig.sty can be obtained at:
% http://www.ctan.org/pkg/subfig




% *** FLOAT PACKAGES ***
%
%\usepackage{fixltx2e}
% fixltx2e, the successor to the earlier fix2col.sty, was written by
% Frank Mittelbach and David Carlisle. This package corrects a few problems
% in the LaTeX2e kernel, the most notable of which is that in current
% LaTeX2e releases, the ordering of single and double column floats is not
% guaranteed to be preserved. Thus, an unpatched LaTeX2e can allow a
% single column figure to be placed prior to an earlier double column
% figure.
% Be aware that LaTeX2e kernels dated 2015 and later have fixltx2e.sty's
% corrections already built into the system in which case a warning will
% be issued if an attempt is made to load fixltx2e.sty as it is no longer
% needed.
% The latest version and documentation can be found at:
% http://www.ctan.org/pkg/fixltx2e


%\usepackage{stfloats}
% stfloats.sty was written by Sigitas Tolusis. This package gives LaTeX2e
% the ability to do double column floats at the bottom of the page as well
% as the top. (e.g., "\begin{figure*}[!b]" is not normally possible in
% LaTeX2e). It also provides a command:
%\fnbelowfloat
% to enable the placement of footnotes below bottom floats (the standard
% LaTeX2e kernel puts them above bottom floats). This is an invasive package
% which rewrites many portions of the LaTeX2e float routines. It may not work
% with other packages that modify the LaTeX2e float routines. The latest
% version and documentation can be obtained at:
% http://www.ctan.org/pkg/stfloats
% Do not use the stfloats baselinefloat ability as the IEEE does not allow
% \baselineskip to stretch. Authors submitting work to the IEEE should note
% that the IEEE rarely uses double column equations and that authors should try
% to avoid such use. Do not be tempted to use the cuted.sty or midfloat.sty
% packages (also by Sigitas Tolusis) as the IEEE does not format its papers in
% such ways.
% Do not attempt to use stfloats with fixltx2e as they are incompatible.
% Instead, use Morten Hogholm'a dblfloatfix which combines the features
% of both fixltx2e and stfloats:
%
% \usepackage{dblfloatfix}
% The latest version can be found at:
% http://www.ctan.org/pkg/dblfloatfix




% *** PDF, URL AND HYPERLINK PACKAGES ***
%
%\usepackage{url}
% url.sty was written by Donald Arseneau. It provides better support for
% handling and breaking URLs. url.sty is already installed on most LaTeX
% systems. The latest version and documentation can be obtained at:
% http://www.ctan.org/pkg/url
% Basically, \url{my_url_here}.




% *** Do not adjust lengths that control margins, column widths, etc. ***
% *** Do not use packages that alter fonts (such as pslatex).         ***
% There should be no need to do such things with IEEEtran.cls V1.6 and later.
% (Unless specifically asked to do so by the journal or conference you plan
% to submit to, of course. )


% correct bad hyphenation here
\hyphenation{op-tical net-works semi-conduc-tor}


\begin{document}
%
% paper title
% Titles are generally capitalized except for words such as a, an, and, as,
% at, but, by, for, in, nor, of, on, or, the, to and up, which are usually
% not capitalized unless they are the first or last word of the title.
% Linebreaks \\ can be used within to get better formatting as desired.
% Do not put math or special symbols in the title.
\title{Convolutional Neural Network \\ Approach to Devise  Controller}


% author names and affiliations
% use a multiple column layout for up to three different
% affiliations
\author{
\IEEEauthorblockN{Yunlong Dong}
\IEEEauthorblockA{School of Automation\\
Huazhong University of Science and Technology
\\
Email: dyl@hust.edu.cn}
\and
\IEEEauthorblockN{Xiangdi Liu}
\IEEEauthorblockA{School of Automation\\
Huazhong University of Science and Technology\\
Email: lxd@hust.edu.cn}
}

% conference papers do not typically use \thanks and this command
% is locked out in conference mode. If really needed, such as for
% the acknowledgment of grants, issue a \IEEEoverridecommandlockouts
% after \documentclass

% for over three affiliations, or if they all won't fit within the width
% of the page, use this alternative format:
%
%\author{\IEEEauthorblockN{Michael Shell\IEEEauthorrefmark{1},
%Homer Simpson\IEEEauthorrefmark{2},
%James Kirk\IEEEauthorrefmark{3},
%Montgomery Scott\IEEEauthorrefmark{3} and
%Eldon Tyrell\IEEEauthorrefmark{4}}
%\IEEEauthorblockA{\IEEEauthorrefmark{1}School of Electrical and Computer Engineering\\
%Georgia Institute of Technology,
%Atlanta, Georgia 30332--0250\\ Email: see http://www.michaelshell.org/contact.html}
%\IEEEauthorblockA{\IEEEauthorrefmark{2}Twentieth Century Fox, Springfield, USA\\
%Email: homer@thesimpsons.com}
%\IEEEauthorblockA{\IEEEauthorrefmark{3}Starfleet Academy, San Francisco, California 96678-2391\\
%Telephone: (800) 555--1212, Fax: (888) 555--1212}
%\IEEEauthorblockA{\IEEEauthorrefmark{4}Tyrell Inc., 123 Replicant Street, Los Angeles, California 90210--4321}}




% use for special paper notices
%\IEEEspecialpapernotice{(Invited Paper)}




% make the title area
\maketitle

% As a general rule, do not put math, special symbols or citations
% in the abstract
\begin{abstract}
PID controller is widely used in many fields. The input sequence and output sequence of a well-parameterized PID controller are transformed into a matrix presented in images, and a Every-Timestep data augmentation algorithm is operated.This paper propose a image to image Convolutional Neural Networks(CNN) structure to learn the PID controller, which firstly incorporating the successful methods in Computer Vision to control field. The simulation is performed using Matlab/Simulink and Keras\cite{chollet2015}. The simulation demonstrates that the CNN controller is capable in performance.
\end{abstract}

% no keywords




% For peer review papers, you can put extra information on the cover
% page as needed:
% \ifCLASSOPTIONpeerreview
% \begin{center} \bfseries EDICS Category: 3-BBND \end{center}
% \fi
%
% For peerreview papers, this IEEEtran command inserts a page break and
% creates the second title. It will be ignored for other modes.
\IEEEpeerreviewmaketitle



\section{Introduction}
In practice PID controller is the mostly used common control algorithm in many fields. PID controller has very simple structure and if parameters tuned well can achiever high performance. Neural Network(NN) was proposed to compose a PID controller\cite{shahraki2009adaptive}\cite{shu2000pid}\cite{shu2005decoupled}.
Neural Network is an information processing structure abstracted by the biological structure, consisting of processing element interconnected together with unidirectional signal channels called connections[1].As the neural network has three advantages, has been widely used in pattern classification, system identification, image processing and other fields. First, Neural Network are data-driven self-adaptive methods. Second, they are universal functional approximators in that they can approximate any function. Third, Neural Networks are powerful nonlinear models, which makes them flexible in handling real world relationships.Over the last few years, machine learning algorithms have led to more efficient methods and traditional neural network has developed different structures for different problems. Cheon\cite{cheon2015replacing} proposed a deep learning method to learn the PID controller.
 
 
 This study combines the images processing in Computer Vision with the control filed, and propose a CNN controller to replace the PID controller. The simulation is performed using Matlab/Simulink and the detail of traditional PID controller and CNN controller was discussed.

This paper is organized as fellows. CNN is described in section \uppercase\expandafter{\romannumeral2}. In section \uppercase\expandafter{\romannumeral3}, Devise Controller is explained. The comparison details between the proposed CNN controller and traditional PID controller are presented with the simulation results are shown in section \uppercase\expandafter{\romannumeral4}. Finally, a conclusion follows in section \uppercase\expandafter{\romannumeral5}.









% An example of a floating figure using the graphicx package.
% Note that \label must occur AFTER (or within) \caption.
% For figures, \caption should occur after the \includegraphics.
% Note that IEEEtran v1.7 and later has special internal code that
% is designed to preserve the operation of \label within \caption
% even when the captionsoff option is in effect. However, because
% of issues like this, it may be the safest practice to put all your
% \label just after \caption rather than within \caption{}.
%
% Reminder: the "draftcls" or "draftclsnofoot", not "draft", class
% option should be used if it is desired that the figures are to be
% displayed while in draft mode.
%
%\begin{figure}[!t]
%\centering
%\includegraphics[width=2.5in]{myfigure}
% where an .eps filename suffix will be assumed under latex,
% and a .pdf suffix will be assumed for pdflatex; or what has been declared
% via \DeclareGraphicsExtensions.
%\caption{Simulation results for the network.}
%\label{fig_sim}
%\end{figure}

% Note that the IEEE typically puts floats only at the top, even when this
% results in a large percentage of a column being occupied by floats.


% An example of a double column floating figure using two subfigures.
% (The subfig.sty package must be loaded for this to work.)
% The subfigure \label commands are set within each subfloat command,
% and the \label for the overall figure must come after \caption.
% \hfil is used as a separator to get equal spacing.
% Watch out that the combined width of all the subfigures on a
% line do not exceed the text width or a line break will occur.
%
%\begin{figure*}[!t]
%\centering
%\subfloat[Case I]{\includegraphics[width=2.5in]{box}%
%\label{fig_first_case}}
%\hfil
%\subfloat[Case II]{\includegraphics[width=2.5in]{box}%
%\label{fig_second_case}}
%\caption{Simulation results for the network.}
%\label{fig_sim}
%\end{figure*}
%
% Note that often IEEE papers with subfigures do not employ subfigure
% captions (using the optional argument to \subfloat[]), but instead will
% reference/describe all of them (a), (b), etc., within the main caption.
% Be aware that for subfig.sty to generate the (a), (b), etc., subfigure
% labels, the optional argument to \subfloat must be present. If a
% subcaption is not desired, just leave its contents blank,
% e.g., \subfloat[].


% An example of a floating table. Note that, for IEEE style tables, the
% \caption command should come BEFORE the table and, given that table
% captions serve much like titles, are usually capitalized except for words
% such as a, an, and, as, at, but, by, for, in, nor, of, on, or, the, to
% and up, which are usually not capitalized unless they are the first or
% last word of the caption. Table text will default to \footnotesize as
% the IEEE normally uses this smaller font for tables.
% The \label must come after \caption as always.
%
%\begin{table}[!t]
%% increase table row spacing, adjust to taste
%\renewcommand{\arraystretch}{1.3}
% if using array.sty, it might be a good idea to tweak the value of
% \extrarowheight as needed to properly center the text within the cells
%\caption{An Example of a Table}
%\label{table_example}
%\centering
%% Some packages, such as MDW tools, offer better commands for making tables
%% than the plain LaTeX2e tabular which is used here.
%\begin{tabular}{|c||c|}
%\hline
%One & Two\\
%\hline
%Three & Four\\
%\hline
%\end{tabular}
%\end{table}


% Note that the IEEE does not put floats in the very first column
% - or typically anywhere on the first page for that matter. Also,
% in-text middle ("here") positioning is typically not used, but it
% is allowed and encouraged for Computer Society conferences (but
% not Computer Society journals). Most IEEE journals/conferences use
% top floats exclusively.
% Note that, LaTeX2e, unlike IEEE journals/conferences, places
% footnotes above bottom floats. This can be corrected via the
% \fnbelowfloat command of the stfloats package.


\section{Convolutional Neural Network}
Convolutional Neural Network(CNN) is a kind of structural variation of traditional neural network,  utilizing layers with convolutional filters that are applied to extract features
. After Alex Krizhevsky used Deep Convolutional Neural Networks in ImageNet classification problem\cite{NIPS2012_4824}, machine learning researcher reach a consensus that general CNN structure composed of convolution layer, pooling layer and fully connected layer.The structure is shown in Fig \ref{fig:cnn}\cite{NIPS2012_4824}.
\begin{figure*}[htbp]
    \centering
     \includegraphics[width=7in]{cnn.png}
     \caption{CNN structure}\label{fig:cnn}
    \end{figure*}


Originally invented for computer version\cite{lecun1995convolutional}, CNN models have subsequently been shown to be effective for many different problems including , Simultaneous Localization and Mapping(SLAM)\cite{naseer2015robust},Decision Making and Automatic Driving\cite{bojarski2016end}.  \\
\indent The Convolutional Neural Network(CNN) plays a significant role in image classification\cite{NIPS2012_4824}, image caption\cite{xu2015show}, image semantic segmentation\cite{long2015fully} and so on. CNN can directly learn from raw images and automatically extract the feature which usually outperform handcrafted feature engineering. CNN uses filters to extract feature from high dimensional sensory data(especially images) and have reached enormous success in many fields.


\section{Devise Controller}
\subsection{Traditional PID controller}
    To devise a novel controller, a traditional PID controller was first proposed to be the baseline. A well-parameterized PID controller was the target we want to mimic.
    The PID controller calculates the output in the following rule(continous time) (\ref{eqn:pid_con}), where the $k_p, k_i, k_d$ are the parameter to be settled in practice, $e$ stands for the error in feedback control loop. The control scheme is showed in Fig \ref{fig:pid} .

    \begin{equation}
    u = k_p \times e + k_i \times \int edt + k_d\times\frac{de}{dt}
    \label{eqn:pid_con}
    \end{equation}

    \begin{figure}[htbp]
     \includegraphics[width=3.5in]{pid.png}
     \caption{PID control scheme}\label{fig:pid}
    \end{figure}

    The PID controller calculates the output based on the error sequence $\{e(0), e(1), e(2), ... , e(T)\}$  in the rule(Discrete time) (\ref{eqn:pid_disc}). Kangbeom Cheon\cite{cheon2015replacing} used a Deep Believe Network to learn the PID controller's input and output, but they do not analysis the underlying principle in PID controller. The PID controller in fact consists of three filter, proportional, integrate, differential. Using the powerful efficiency of Deep learning methods, the map between PID controller input and output can be easily learned via supervising learning. The learning scheme is shown in Fig \ref{fig:learn_scheme}.

    \begin{equation}
    \begin{split}
    u(k) = k_p \times e(k) + k_i \times \Delta t \sum_{i=0}^{K}e(i) + k_d\times \frac{e(k) - e(k-1)}{\Delta t}
    \label{eqn:pid_disc}
    \end{split}
    \end{equation}




    \begin{figure}[htbp]
     \includegraphics[width=3.5in]{learning_scheme.png}
     \caption{learning scheme}\label{fig:learn_scheme}
    \end{figure}


 \subsection{Transfer to Image}
 The error sequence $\{e(0), e(1), e(2), ... , e(T)\}$ is the input of our learning algorithm(Deep learning methods). Instead of directly dropping the Sequence to the model, We analysis the sequence first. The sequence is obviously as time sequence which implies the system response in time domain. The RNN[???] like models are capable of dealing with such sequence. But the time
sequence is just a One-Dimensional relationship, we want further exploration. So we transform the error sequence to a matrix which has a spatial structure, which can contain more information. We samples 20s time(the default terminated time) in error with sample time 0.01s to get a 2000 length sequence. Then the Sequence is transformed into a $40\times50$ matrix called \textbf{error plane}. The same process was done in the output of PID controller called \textbf{u plane}. For visualization convenience, The matrix(One channel) are shown with colorful format in Fig \ref{fig:error_u}.
    \begin{figure}[htbp]
    \centering
     \includegraphics[width=1.5in]{error_plane.png}
     \includegraphics[width=1.5in]{u_plane.png}
     \caption{error plane(left), u plane(right)}\label{fig:error_u}
    \end{figure}

As can be clearly seen, there is a pattern in the error plane and u plane. The u plane is like the blurry error plane. We can treat the sequence like image. There are many well-developed theory, models and algorithm processing images[???]. Now we transfer the learning controller problem to a image to image supervising learning problem.

\begin{figure*}[htbp]

    \includegraphics[width=7in]{cnn_pid.png}
    % where an .eps filename suffix will be assumed under latex,
    % and a .pdf suffix will be assumed for pdflatex; or what has been declared
    % via \DeclareGraphicsExtensions.

    \caption{The CNN controller architecture.}
    \label{fig:cnn_pid}

 \end{figure*}

\subsection{Devise CNN controller}



The architecture used to devise the controller is shown in Fig \ref{fig:cnn_pid}. The architecture has  purely convolutional layers. The output of the CNN controller should have the same size with the input, so there no no pooling layers and Flattern layers . The first layer has 32 $3\times3$  filters with relu activation function, the second layer has the same filters as the first layer, and the last layer have 1 $3\times3$ filter with relu activation function to make the output of CNN has size $40\times50$.







Two-Dimentional convolutional operation can be written in eqn (\ref{eqn:conv}), where $K$ is the filter with size $k_r\times k_c$, $In(s, t), out(s, t)$ are the input and output of the convolutional operation with size $N_c\times N_r$ respectively. Through convolutional operation, the CNN is able to capture the relationship and maps between error plane and u plane. Semantic, we want the filter to learn the pid caclulating rule eqn (\ref{eqn:pid_disc}).



\begin{equation}
\begin{split}
    Out(s, t) &= \sum_{m=0}^{k_r-1} \sum_{n=0}^{k_c -1} K(m, n) \times In(s-m, t-n) \\
    s.t.\ 0 &\leq s \leq N_r-1,\ 0 \leq t \leq N-c -1
    \label{eqn:conv}
\end{split}
\end{equation}





\subsection{Training}
In practice the whole length of error sequence is not able to obtain, cause the length of error increases as the controlling time passing. In general, in time step $k, k< 2000$, the error sequence is $\{(e(0), e(1), e(2), ... ,e(k))\}$. We fullfill the sequence to length of 2000 with zeros to obtain $\{(e(0), e(1), e(2), ... ,e(k), 0, 0, ...)\}$.
In training we only gather the full-length error plane and u plane, in other words, only one pair of data is available. In controlling loop, every time step should obey the rule(???), so the error plane and u plane can be split in every time step to generate a called \textbf{Every-Timestep error-u} dataset. The detail goes in \textbf{Algorithm 1}. $error, u$ are the error and u sequence with length $T$(default terminated time).
Once the augmented dataset has been obtained. A mini-batch Stochastic Gradient Descent opimizer with $lerning\_rate = 0.1, batch\_size=32, momentum=0.9, epochs = 10$ is used to train the CNN. Keras \cite{chollet2015} was used to achieve these stuff. The results are shown in Fig \ref{fig:train_test}. We use $70\%$ of dataset to train and the rest $30\%$ to test, and proving that overfitting did not appear. The visualization of training peroformance is shown in Fig \ref{fig:performance}.
\begin{algorithm}[!h]
    \caption{Every-Timestep$(error,u, T)$}%�㷨����
    \begin{algorithmic}[1]%һ��һ�����к�
        \STATE $error\_gen = Zeros[T, T]$
        \STATE $u\_gen = Zeros[T, T]$
        \FOR{$i=1$ to $T$}
            \STATE{$error\_gen[i, 1:i] = error[1:i]$}
            \STATE{$u\_gen[i, 1:i] = u[1:i]$}
        \ENDFOR
        \RETURN {$\{error\_gen, u\_gen\}$}
    \end{algorithmic}
\end{algorithm}

\begin{figure}[htbp]
    \centering
     \includegraphics[width=3.5in]{train_test.png}

     \caption{training results}\label{fig:train_test}
    \end{figure}
\begin{figure}[htbp]
    \centering
     \includegraphics[width=3.5in]{performance.png}

     \caption{training visualization}\label{fig:performance}
    \end{figure}



\section{Evaluation}

In evaluation, we first parameterized a traditional PID controller in Simulink to simulate. The closed loop is shown in Fig \ref{fig:control}. Then we replace the PID controller with our proposed CNN controller. Simulating in 2s. The comparing results of step response are shown in Fig \ref{fig:step}.
\begin{figure}[htbp]
    \centering
     \includegraphics[width=3.6in]{control.png}

     \caption{simulink model}\label{fig:control}
    \end{figure}
\begin{figure}[!h]
    \centering
     \includegraphics[width=3.6in]{step.png}

     \caption{simulink model}\label{fig:step}
    \end{figure}
The CNN controller perform almost the same with PID controller even a little more smooth.
\section{Conclusion}
In this paper, a Convolutional Neuron Networks controller was devised as an approach to replace the PID controller. So far this paper is the first to combine images with control, incorporating the successful mothods in Computer Vision to control field, and propose a image to image CNN to learn the controller. The simulation demonstrates that the CNN controller is capable. We only apply the most used architecture CNN in this control problem, it is believing that there must be more architecture widely used in Computer Vision suitable to play a significant role in control field.




% conference papers do not normally have an appendix


% use section* for acknowledgment
\section*{Acknowledgment}

This research was supported by Ye Yuan, Professor of Control Theory, School of Automation,
State Key Lab of Digital Manufacturing Equipment and Technology,
Huazhong University of Science and Technology.





% trigger a \newpage just before the given reference
% number - used to balance the columns on the last page
% adjust value as needed - may need to be readjusted if
% the document is modified later
%\IEEEtriggeratref{8}
% The "triggered" command can be changed if desired:
%\IEEEtriggercmd{\enlargethispage{-5in}}

% references section

% can use a bibliography generated by BibTeX as a .bbl file
% BibTeX documentation can be easily obtained at:
% http://mirror.ctan.org/biblio/bibtex/contrib/doc/
% The IEEEtran BibTeX style support page is at:
% http://www.michaelshell.org/tex/ieeetran/bibtex/
%\bibliographystyle{IEEEtran}
% argument is your BibTeX string definitions and bibliography database(s)
%\bibliography{IEEEabrv,../bib/paper}
%
% <OR> manually copy in the resultant .bbl file
% set second argument of \begin to the number of references
% (used to reserve space for the reference number labels box)





\bibliography{bibfile}{}
\bibliographystyle{plain}







% that's all folks
\end{document}


